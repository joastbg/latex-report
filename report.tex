\documentclass[a4paper,11pt]{report}
%
%--------------------   start of the 'preamble'
%
\usepackage{graphicx,amssymb,amstext,amsmath,graphics,epsfig,color}
\usepackage{fancyhdr}
\usepackage{algorithm}
\usepackage{algorithmic}
%
%%    homebrew commands -- to save typing
\newcommand\etc{\textsl{etc}}
\newcommand\eg{\textsl{eg.}\ }
\newcommand\etal{\textsl{et al.}}
\newcommand\Quote[1]{\lq\textsl{#1}\rq}
\newcommand\fr[2]{{\textstyle\frac{#1}{#2}}}
\newcommand\miktex{\textsl{MikTeX}}
\newcommand\comp{\textsl{The Companion}}
\newcommand\nss{\textsl{Not so Short}}
%
%---------------------   end of the 'preamble'
%
\newcommand{\HRule}{\rule{\linewidth}{0.5mm}}

\pagestyle{fancy}               % Fräcka sidhuvuden
\addtolength{\headwidth}{2cm}   % Sidhuvd bredare än texten.
\renewcommand{\headrulewidth}{0.4pt} % Linje i sidhuvud är 0.4 punkter
%\renewcommand{\footrulewidth}{0.4pt} % Linje i sidfot är 2 punkter

% Följande kommandon definerar vad som ska finnas i sidhuvud och
% sidfot. Om man skriver dubbelsidiga dokument anger man två alternativ
% med komma mellan. Den första gäller då för udda sidor och den andra
% för jämna sidor. Bokstäverna ska tolkas som:
% L = left, C = center, R = right,
% E = even (jämna sidor), O = odd (udda sidor)
% E och O fyller ingen funktion om man inte har optionen twopage definierad
\fancyhead[R]{\bf{\nouppercase{\leftmark}}}	% Vänstertext i sidhuvud
\fancyhead[L]{\nouppercase{\rightmark}}	% Högertext i sidhuvud
%\fancyfoot[C]{}	% Mittentext i sidfot
%\fancyfoot[LO,RE]{}		% Vänster udda, höger jämna sidor
%\fancyfoot[LE,RO]{}	% Vänster jämna, höger udda sidor


\begin{document}
%-----------------------------------------------------------
\begin{titlepage}

\textsc{\LARGE }\\[1.5cm]
\textsc{\Large }\\[0.5cm]
\textsc{\large }\\[0.5cm]
 
% Title
\begin{center}
\HRule \\[0.4cm]
{ \huge \bfseries Implementation of a modern\\ ray tracer}\\[0.4cm]
\HRule \\[1.5cm]

% Author
\begin{center} \Large
\emph{Author:}\\
\textsc{Johan Astborg}\\[3cm]
\end{center}

% Bottom of the page
{\large January 25, 2011}\\[4cm]
%\includegraphics{Logo}\\[1cm] % Department/University logo
 
\vfill
\end{center}

\end{titlepage}
%-----------------------------------------------------------
\begin{abstract}\centering

The building of a multithreaded, high performance ray tracer supporting
direct and global illumination, shading and meshes.
\end{abstract}
%-----------------------------------------------------------
\tableofcontents
%-----------------------------------------------------------
\listoftables
\addcontentsline{toc}{chapter}{List of Figures}
\listoffigures

\chapter{Introduction}
This document will describe in detail the building blocks of a modern high performance ray tracer.
Feature list:
\begin{itemize}
\item Choice of Language, C++
\item Basic Whitted ray-tracing
\item Kd-Tree acceleration structure
\end{itemize}
Cras nisi neque, pharetra ac cursus nec, vestibulum sit amet erat. Vivamus eget viverra elit. Sed vehicula augue sit amet nibh convallis volutpat. Sed feugiat posuere nunc a auctor. Nam turpis erat, ultrices sed varius in, tempus nec enim. Donec hendrerit dignissim libero, non lacinia odio congue non. Nulla eu velit urna, ut accumsan nibh. Fusce ligula massa, volutpat ut blandit at, dignissim sed orci. Nulla sed mauris lorem. Mauris nec turpis purus, sed sollicitudin massa. Sed ipsum purus, vestibulum et viverra et, tristique at leo. Cum sociis natoque penatibus et magnis dis parturient montes, nascetur ridiculus mus. Sed gravida, odio a rutrum posuere, diam erat fermentum arcu, sit amet blandit orci metus ac mauris.

\chapter{Fundamentals}
Lorem ipsum dolor sit amet, consectetur adipiscing elit. Donec semper accumsan velit, eget consequat tortor aliquam vitae. Nunc ut nisi metus. 
Pellentesque risus ipsum, facilisis vel porttitor accumsan, volutpat a massa. Nam id lorem tortor, et mollis felis. Suspendisse potenti. 
Nunc a turpis ac ipsum cursus malesuada sed quis elit. Sed ante diam, porttitor id consectetur a, condimentum sed purus. 
\\\\
Mauris ultricies justo eget ante hendrerit sed auctor lacus lobortis. Curabitur turpis erat, suscipit at ullamcorper ut, fringilla non metus. 
Cras porttitor ipsum quis libero interdum quis gravida magna sodales. Fusce euismod, tellus vitae porta dictum, nunc tellus tempus turpis, 
a elementum justo magna vitae metus. Mauris erat felis, posuere bibendum malesuada sit amet, tincidunt vel est. Suspendisse laoreet ullamcorper 
ligula, et molestie orci ullamcorper ut. Praesent mollis sem a nibh condimentum adipiscing. Aenean vitae magna magna. 
Praesent quis lorem ipsum, eget lobortis tortor whitted\cite{Whitted1979}.
\section{The coordinate system}
\begin{figure}[htbp]
\centering
\input{lefthanded.pstex_t}
\caption{Left-handed coordinate system}
\label{Left-handed cartesian coordinate system.}
\end{figure}
\begin{itemize}
\item Cartesian coordinate system
\item Right handed and left handed
\item Right handed (+Z out of screen)
\item Left handed (+Z into the screen)
\end{itemize}
\section{Point}
Ut in erat tellus. Vestibulum ante ipsum primis in faucibus orci luctus et ultrices posuere cubilia Curae; Integer id fringilla massa. Praesent ut urna vel neque eleifend congue. Donec libero felis, placerat ac scelerisque sed, eleifend ac nisl. Nullam sapien odio, volutpat at venenatis nec, sollicitudin condimentum magna. In hac habitasse platea dictumst. Integer enim turpis, pellentesque in commodo ac, sagittis lobortis erat.
\section{Vector}
Ut in erat tellus. Vestibulum ante ipsum primis in faucibus orci luctus et ultrices posuere cubilia Curae; Integer id fringilla massa. Praesent ut urna vel neque eleifend congue. Donec libero felis, placerat ac scelerisque sed, eleifend ac nisl. Nullam sapien odio, volutpat at venenatis nec, sollicitudin condimentum magna. In hac habitasse platea dictumst. Integer enim turpis, pellentesque in commodo ac, sagittis lobortis erat.
\section{Ray}
Ut in erat tellus. Vestibulum ante ipsum primis in faucibus orci luctus et ultrices posuere cubilia Curae; Integer id fringilla massa. Praesent ut urna vel neque eleifend congue. Donec libero felis, placerat ac scelerisque sed, eleifend ac nisl. Nullam sapien odio, volutpat at venenatis nec, sollicitudin condimentum magna. In hac habitasse platea dictumst. Integer enim turpis, pellentesque in commodo ac, sagittis lobortis erat.
\section{Matrix}
Ut in erat tellus. Vestibulum ante ipsum primis in faucibus orci luctus et ultrices posuere cubilia Curae; Integer id fringilla massa. Praesent ut urna vel neque eleifend congue. Donec libero felis, placerat ac scelerisque sed, eleifend ac nisl. Nullam sapien odio, volutpat at venenatis nec, sollicitudin condimentum magna. In hac habitasse platea dictumst. Integer enim turpis, pellentesque in commodo ac, sagittis lobortis erat.
\section{Plane primitive}
Ut in erat tellus. Vestibulum ante ipsum primis in faucibus orci luctus et ultrices posuere cubilia Curae; Integer id fringilla massa. Praesent ut urna vel neque eleifend congue. Donec libero felis, placerat ac scelerisque sed, eleifend ac nisl. Nullam sapien odio, volutpat at venenatis nec, sollicitudin condimentum magna. In hac habitasse platea dictumst. Integer enim turpis, pellentesque in commodo ac, sagittis lobortis erat.
\section{Plane intersection}
Ut in erat tellus. Vestibulum ante ipsum primis in faucibus orci luctus et ultrices posuere cubilia Curae; Integer id fringilla massa. Praesent ut urna vel neque eleifend congue. Donec libero felis, placerat ac scelerisque sed, eleifend ac nisl. Nullam sapien odio, volutpat at venenatis nec, sollicitudin condimentum magna. In hac habitasse platea dictumst. Integer enim turpis, pellentesque in commodo ac, sagittis lobortis erat.
\section{Box primitive}
Ut in erat tellus. Vestibulum ante ipsum primis in faucibus orci luctus et ultrices posuere cubilia Curae; Integer id fringilla massa. Praesent ut urna vel neque eleifend congue. Donec libero felis, placerat ac scelerisque sed, eleifend ac nisl. Nullam sapien odio, volutpat at venenatis nec, sollicitudin condimentum magna. In hac habitasse platea dictumst. Integer enim turpis, pellentesque in commodo ac, sagittis lobortis erat.
\section{Box intersection}
Ut in erat tellus. Vestibulum ante ipsum primis in faucibus orci luctus et ultrices posuere cubilia Curae; Integer id fringilla massa. Praesent ut urna vel neque eleifend congue. Donec libero felis, placerat ac scelerisque sed, eleifend ac nisl. Nullam sapien odio, volutpat at venenatis nec, sollicitudin condimentum magna. In hac habitasse platea dictumst. Integer enim turpis, pellentesque in commodo ac, sagittis lobortis erat.
\section{Triangle primitive}
Ut in erat tellus. Vestibulum ante ipsum primis in faucibus orci luctus et ultrices posuere cubilia Curae; Integer id fringilla massa. Praesent ut urna vel neque eleifend congue. Donec libero felis, placerat ac scelerisque sed, eleifend ac nisl. Nullam sapien odio, volutpat at venenatis nec, sollicitudin condimentum magna. In hac habitasse platea dictumst. Integer enim turpis, pellentesque in commodo ac, sagittis lobortis erat.
\section{Triangle intersection}
Ut in erat tellus. Vestibulum ante ipsum primis in faucibus orci luctus et ultrices posuere cubilia Curae; Integer id fringilla massa. Praesent ut urna vel neque eleifend congue. Donec libero felis, placerat ac scelerisque sed, eleifend ac nisl. Nullam sapien odio, volutpat at venenatis nec, sollicitudin condimentum magna. In hac habitasse platea dictumst. Integer enim turpis, pellentesque in commodo ac, sagittis lobortis erat.

\chapter{Ray tracing}
Pellentesque felis nulla, interdum eu pretium vitae, tincidunt ultrices erat. Integer a quam ut nibh lobortis vulputate. Phasellus felis augue, sagittis vitae tempus porta, placerat ut dui. Quisque commodo, erat ut iaculis viverra, urna dui luctus erat, ac rutrum ligula mauris non arcu. Maecenas nisi nunc, pulvinar in interdum eget, adipiscing quis lorem. Etiam hendrerit faucibus orci, ac elementum dui posuere eget. Vivamus felis purus, eleifend non tempus sed, dapibus in risus. Suspendisse et ipsum augue.
\section{The scene}
Ut in erat tellus. Vestibulum ante ipsum primis in faucibus orci luctus et ultrices posuere cubilia Curae; Integer id fringilla massa. Praesent ut urna vel neque eleifend congue. Donec libero felis, placerat ac scelerisque sed, eleifend ac nisl. Nullam sapien odio, volutpat at venenatis nec, sollicitudin condimentum magna. In hac habitasse platea dictumst. Integer enim turpis, pellentesque in commodo ac, sagittis lobortis erat.
\section{Light sources}
Ut in erat tellus. Vestibulum ante ipsum primis in faucibus orci luctus et ultrices posuere cubilia Curae; Integer id fringilla massa. Praesent ut urna vel neque eleifend congue. Donec libero felis, placerat ac scelerisque sed, eleifend ac nisl. Nullam sapien odio, volutpat at venenatis nec, sollicitudin condimentum magna. In hac habitasse platea dictumst. Integer enim turpis, pellentesque in commodo ac, sagittis lobortis erat.
\section{Ray intersection}
Ut in erat tellus. Vestibulum ante ipsum primis in faucibus orci luctus et ultrices posuere cubilia Curae; Integer id fringilla massa. Praesent ut urna vel neque eleifend congue. Donec libero felis, placerat ac scelerisque sed, eleifend ac nisl. Nullam sapien odio, volutpat at venenatis nec, sollicitudin condimentum magna. In hac habitasse platea dictumst. Integer enim turpis, pellentesque in commodo ac, sagittis lobortis erat.
\section{Shading}
Ut in erat tellus. Vestibulum ante ipsum primis in faucibus orci luctus et ultrices posuere cubilia Curae; Integer id fringilla massa. Praesent ut urna vel neque eleifend congue. Donec libero felis, placerat ac scelerisque sed, eleifend ac nisl. Nullam sapien odio, volutpat at venenatis nec, sollicitudin condimentum magna. In hac habitasse platea dictumst. Integer enim turpis, pellentesque in commodo ac, sagittis lobortis erat.
\section{Reflection}
The reflection direction can be found using the surface normal $n$ and the ray direction $d$:\\
\[
\mathbf r = \mathbf d - 2(\mathbf n \cdot \mathbf d ) \mathbf n.
\]
The reflected ray has equation:
\[
\mathbf x = \mathbf y  + u \mathbf r.
\]
\section{Refraction}
Ut in erat tellus. Vestibulum ante ipsum primis in faucibus orci luctus et ultrices posuere cubilia Curae; Integer id fringilla massa. Praesent ut urna vel neque eleifend congue. Donec libero felis, placerat ac scelerisque sed, eleifend ac nisl. Nullam sapien odio, volutpat at venenatis nec, sollicitudin condimentum magna. In hac habitasse platea dictumst. Integer enim turpis, pellentesque in commodo ac, sagittis lobortis erat.
\\
\begin{table}
%\centering
\begin{tabular}{| l | c | r |}
\hline
Material          			& P \\
\hline
Ice 						& 1.309 \\
Water ($0^{\circ}$			& 1.33346 \\
Water ($20^{\circ}$			& 1.33283 \\
Flint glass (29\% lead) 	& 1.569 \\
Flint glass (55\% lead) 	& 1.669 \\
Flint glass (71\% lead) 	& 1.805 \\
Crown glass					& 1.52 \\
Air 						& 1.000293 \\
Diamond 					& 2.42 \\
Ethanol 					& 1.36 \\
Methanol 					& 1.329 \\
\hline
\end{tabular}
\caption{Refraction indices}
\label{tab:Refraction indices}
\end{table}

\section{Shadows}
Penumbra
\section{Anti-aliasing}
Oversampling

\chapter{The camera}
Ut in erat tellus. Vestibulum ante ipsum primis in faucibus orci luctus et ultrices posuere cubilia Curae; Integer id fringilla massa. Praesent ut urna vel neque eleifend congue. Donec libero felis, placerat ac scelerisque sed, eleifend ac nisl. Nullam sapien odio, volutpat at venenatis nec, sollicitudin condimentum magna. In hac habitasse platea dictumst. Integer enim turpis, pellentesque in commodo ac, sagittis lobortis erat.
\section{A FOV-based camera system}
Lorem ipsum dolor sit amet, consectetur adipiscing elit. Donec semper accumsan velit, eget consequat tortor aliquam vitae. Nunc ut nisi metus. Pellentesque risus ipsum, facilisis vel porttitor accumsan, volutpat a massa. Nam id lorem tortor, et mollis felis. Suspendisse potenti. Nunc a turpis ac ipsum cursus malesuada sed quis elit. Sed ante diam, porttitor id consectetur a, condimentum sed purus. Mauris ultricies justo eget ante hendrerit sed auctor lacus lobortis. Curabitur turpis erat, suscipit at ullamcorper ut, fringilla non metus. Cras porttitor ipsum quis libero interdum quis gravida magna sodales. Fusce euismod, tellus vitae porta dictum, nunc tellus tempus turpis, a elementum justo magna vitae metus. Mauris erat felis, posuere bibendum malesuada sit amet, tincidunt vel est. Suspendisse laoreet ullamcorper ligula, et molestie orci ullamcorper ut. Praesent mollis sem a nibh condimentum adipiscing. Aenean vitae magna magna. Praesent quis lorem ipsum, eget lobortis tortor.
\subsection{Perspective camera}
Class aptent taciti sociosqu ad litora torquent per conubia nostra, per inceptos himenaeos. Quisque at arcu risus. Vestibulum ullamcorper quam ut ipsum elementum porta. Proin et lorem nibh, vel vulputate justo.
\subsection{The focal-plane}
Donec libero felis, placerat ac scelerisque sed, eleifend ac nisl. Nullam sapien odio, volutpat at venenatis nec, sollicitudin condimentum magna. In hac habitasse platea dictumst. Integer enim turpis, pellentesque in commodo ac, sagittis lobortis erat.
\subsection{Camera and world coordinates}
Class aptent taciti sociosqu ad litora torquent per conubia nostra, per inceptos himenaeos. Quisque at arcu risus. Vestibulum ullamcorper quam ut ipsum elementum porta. Proin et lorem nibh, vel vulputate justo.
\subsection{The camera matrix}
Class aptent taciti sociosqu ad litora torquent per conubia nostra, per inceptos himenaeos. Quisque at arcu risus. Vestibulum ullamcorper quam ut ipsum elementum porta. Proin et lorem nibh, vel vulputate justo.
\[
\begin{bmatrix}
   -f/S_x		 	& 0 		& n_0 	\\
   0 				& -f/S_y    & m_0	\\
   0           		& 0 		& 1
\end{bmatrix}
\begin{bmatrix}
	- \hat{u}^\mathrm{T} -	\\
	- \hat{v}^\mathrm{T} -	\\
	- \hat{p}^\mathrm{T} -
\end{bmatrix}
\begin{bmatrix}
	1 & 0 & 0 & -C_x	\\
	0 & 1 & 0 & -C_y	\\
	0 & 0 & 1 & -C_y	\\
\end{bmatrix}
\begin{bmatrix}
	P_x \\
	P_y \\
	P_z	\\
	1
\end{bmatrix}
=
\begin{bmatrix}
	\alpha n	\\
	\alpha m	\\
	\alpha
\end{bmatrix}
\]

Solve for ray position column vector and set $\alpha$ = f = 1
\[
\begin{bmatrix}
	P_x \\
	P_y \\
	P_z	\\
	1
\end{bmatrix}
=
\begin{bmatrix}
	\alpha n	\\
	\alpha m	\\
	\alpha
\end{bmatrix}
\begin{bmatrix}
   -f/S_x		 	& 0 		& n_0 	\\
   0 				& -f/S_y    & m_0	\\
   0           		& 0 		& 1
\end{bmatrix}^{-1}
\begin{bmatrix}
	- \hat{u}^\mathrm{T} -	\\
	- \hat{v}^\mathrm{T} -	\\
	- \hat{p}^\mathrm{T} -
\end{bmatrix}^{-1}
\begin{bmatrix}
	1 & 0 & 0 & -C_x	\\
	0 & 1 & 0 & -C_y	\\
	0 & 0 & 1 & -C_y	\\
\end{bmatrix}^{-1}
\]
\[
\iff
\]
\[
\begin{bmatrix}
	P_x \\
	P_y \\
	P_z	\\
	1
\end{bmatrix}
=
\begin{bmatrix}
	n	\\
	m	\\
	1
\end{bmatrix}
\underbrace{
\begin{bmatrix}
   -1/S_x		 	& 0 		& n_0 	\\
   0 				& -1/S_y    & m_0	\\
   0           		& 0 		& 1
\end{bmatrix}^{-1}
\begin{bmatrix}
	- \hat{u}^\mathrm{T} -	\\
	- \hat{v}^\mathrm{T} -	\\
	- \hat{p}^\mathrm{T} -
\end{bmatrix}^{-1}
\begin{bmatrix}
	1 & 0 & 0 & -C_x	\\
	0 & 1 & 0 & -C_y	\\
	0 & 0 & 1 & -C_y	\\
\end{bmatrix}^{-1}
}_{\mbox{constant, i.e. can be precalculated}}
\]
\cite{Borman03raytracingand}

\chapter{Shading}
Donec libero felis, placerat ac scelerisque sed, eleifend ac nisl. Nullam sapien odio, volutpat at venenatis nec, sollicitudin condimentum magna. In hac habitasse platea dictumst. Integer enim turpis, pellentesque in commodo ac, sagittis lobortis erat.
\section{Turner shading model}
Pellentesque felis nulla, interdum eu pretium vitae, tincidunt ultrices erat. Integer a quam ut nibh lobortis vulputate. Phasellus felis augue, sagittis vitae tempus porta, placerat ut dui. Quisque commodo, erat ut iaculis viverra, urna dui luctus erat, ac rutrum ligula mauris non arcu. Maecenas nisi nunc, pulvinar in interdum eget, adipiscing quis lorem. Etiam hendrerit faucibus orci, ac elementum dui posuere eget. Vivamus felis purus, eleifend non tempus sed, dapibus in risus. Suspendisse et ipsum augue.
\subsection{Ambient}
Class aptent taciti sociosqu ad litora torquent per conubia nostra, per inceptos himenaeos. Quisque at arcu risus. Vestibulum ullamcorper quam ut ipsum elementum porta. Proin et lorem nibh, vel vulputate justo. Nulla facilisi. Morbi diam mi, eleifend et euismod eget, blandit ut mi. Cras et convallis erat. 
\subsection{Diffuse}
Ut in erat tellus. Vestibulum ante ipsum primis in faucibus orci luctus et ultrices posuere cubilia Curae; Integer id fringilla massa. Praesent ut urna vel neque eleifend congue. 
\subsection{Specular}
Lorem ipsum dolor sit amet, consectetur adipiscing elit. Donec semper accumsan velit, eget consequat tortor aliquam vitae. Nunc ut nisi metus. Pellentesque risus ipsum, facilisis vel porttitor accumsan, volutpat a massa. Nam id lorem tortor, et mollis felis.
\section{Phong}
Class aptent taciti sociosqu ad litora torquent per conubia nostra, per inceptos himenaeos. Quisque at arcu risus. Vestibulum ullamcorper quam ut ipsum elementum porta. Proin et lorem nibh, vel vulputate justo. Nulla facilisi. Morbi diam mi, eleifend et euismod eget, blandit ut mi. Cras et convallis erat. 
\section{BRDF}
Bidirectional reflectance distribution function:
$f_r(\omega_i, \omega_o) = \dfrac{dL_r(\omega_o)}{L_i(\omega_i)\cos{\theta_i}d\omega_i}$
\begin{itemize}
\item 4 dimentions
\item Ratio between amount of reflected light in direction w, and the amount received from direction w' 
\end{itemize}
\section{Anisotropic reflection}
Lorem ipsum dolor sit amet, consectetur adipiscing elit. Donec semper accumsan velit, eget consequat tortor aliquam vitae. Nunc ut nisi metus. Pellentesque risus ipsum, facilisis vel porttitor accumsan, volutpat a massa. Nam id lorem tortor, et mollis felis.
\section{Scattering}
Class aptent taciti sociosqu ad litora torquent per conubia nostra, per inceptos himenaeos. Quisque at arcu risus. Vestibulum ullamcorper quam ut ipsum elementum porta. Proin et lorem nibh, vel vulputate justo. Nulla facilisi. Morbi diam mi, eleifend et euismod eget, blandit ut mi. Cras et convallis erat. 
\subsection{Lambertian}
Class aptent taciti sociosqu ad litora torquent per conubia nostra, per inceptos himenaeos. Quisque at arcu risus. Vestibulum ullamcorper quam ut ipsum elementum porta. Proin et lorem nibh, vel vulputate justo. Nulla facilisi. Morbi diam mi, eleifend et euismod eget, blandit ut mi. Cras et convallis erat. 
\subsection{Schlick model}
Class aptent taciti sociosqu ad litora torquent per conubia nostra, per inceptos himenaeos. Quisque at arcu risus. Vestibulum ullamcorper quam ut ipsum elementum porta. Proin et lorem nibh, vel vulputate justo. Nulla facilisi. Morbi diam mi, eleifend et euismod eget, blandit ut mi. Cras et convallis erat. 
\subsection{Ward model}
Class aptent taciti sociosqu ad litora torquent per conubia nostra, per inceptos himenaeos. Quisque at arcu risus. Vestibulum ullamcorper quam ut ipsum elementum porta. Proin et lorem nibh, vel vulputate justo. Nulla facilisi. Morbi diam mi, eleifend et euismod eget, blandit ut mi. Cras et convallis erat. 
\subsection{Ashikhmin \& Shirley}
Class aptent taciti sociosqu ad litora torquent per conubia nostra, per inceptos himenaeos. Quisque at arcu risus. Vestibulum ullamcorper quam ut ipsum elementum porta. Proin et lorem nibh, vel vulputate justo. Nulla facilisi. Morbi diam mi, eleifend et euismod eget, blandit ut mi. Cras et convallis erat. 
\section{Bumpmapping}
Class aptent taciti sociosqu ad litora torquent per conubia nostra, per inceptos himenaeos. Quisque at arcu risus. Vestibulum ullamcorper quam ut ipsum elementum porta. Proin et lorem nibh, vel vulputate justo. Nulla facilisi. Morbi diam mi, eleifend et euismod eget, blandit ut mi. Cras et convallis erat. 
\chapter{Global illumination}
\section{Overview}
Compare path tracing and photon mapping in aspects:
\begin{itemize}
\item Rendering image quality
\item Rendering speed
\item Physical phenomenons supported
\item Progressive rendering or not
\item GPU support
\item Optimization friendly
\item Large scenes, ie scalability
\item Memory performance and usage
\item Scalability threads and multiprocessor systems
\end{itemize}
\subsection{Path tracing}
\begin{itemize}
\item Variance shows up as noise
\item Computationally intensive
\item Probabilistic sampling technique
\item Random process
\item Markov chain
\item Kajiya developed path tracing as a solution to the rendering the equation
\item Indirect lighting
\item Caustics
\item Tracing a single path from the eye back to the light source
\item Non biased method
\item Slow, ie computationally expencive
\item Monte Carlo Method
\item Importance Sampling
\item Irradiance Caching
\item Direct Light
\end{itemize}
\subsection{Photon mapping}
\begin{itemize}
\item Flexible, render equation split up in passes
\item Photons emitted, result stored in photon map
\item Solves the rendering equation, some bias introduces
\item Rays from light source
\item Rays from camera
\end{itemize}
\section{The rendering equation}
\[
L_o(\mathbf x, \omega, \lambda, t) = L_e(\mathbf x, \omega, \lambda, t) + \int_\Omega f_r(\mathbf x, \omega', \omega, \lambda, t) L_i(\mathbf x, \omega', \lambda, t) (-\omega' \cdot \mathbf n) d \omega'
\]
We can simplify the above and reduce variable $\lambda$ and $t$, constant time and constant wavelength (per component vector math). The $L_e$ term can also be omitted, because only light sources are emitters.
\[
L_o(\mathbf x, \omega) = \int_\Omega f_r(\mathbf x, \omega', \omega) L_i(\mathbf x, \omega') (-\omega' \cdot \mathbf n) d \omega'
\]

\begin{itemize}
\item Finite element methods and point sampling techniques
\item Rendering is the process of solving nested integrals
\item No analytical solution to integrals
\end{itemize}
\subsection{Background}
simultaneously introduced into computer graphics by David Immel et al and James Kajiya in 1986.
\subsection{Deriving the rendering equation}
Reflection function.
\section{Photon mapping in detail}
Photon mapping is the prefered way of approximating the rendering equation, considering both performance and the high quality result it provides.
The algorithm also supports caustics. Henrik Wann Jensen invented Photon Mapping as a faster alternative to Monte Carlo integrators.
\subsection{Approximating the rendering equation}
Cras nisi neque, pharetra ac cursus nec, vestibulum sit amet erat. Vivamus eget viverra elit. Sed vehicula augue sit amet nibh convallis volutpat. Sed feugiat posuere nunc a auctor. Nam turpis erat, ultrices sed varius in, tempus nec enim. Donec hendrerit dignissim libero, non lacinia odio congue non. 
\subsection{Pass I: Construction of the photon map}
Cras nisi neque, pharetra ac cursus nec, vestibulum sit amet erat. Vivamus eget viverra elit. Sed vehicula augue sit amet nibh convallis volutpat. Sed feugiat posuere nunc a auctor. Nam turpis erat, ultrices sed varius in, tempus nec enim. Donec hendrerit dignissim libero, non lacinia odio congue non. 
\subsection{Pass II: Rendering}
Cras nisi neque, pharetra ac cursus nec, vestibulum sit amet erat. Vivamus eget viverra elit. Sed vehicula augue sit amet nibh convallis volutpat. Sed feugiat posuere nunc a auctor. Nam turpis erat, ultrices sed varius in, tempus nec enim. Donec hendrerit dignissim libero, non lacinia odio congue non. 
\subsection{Irradiance caching}
\begin{itemize}
\item Lazy evaluation
\item Greg Ward '88
\item Cache entry contains the position, normal and irradiance value itself
\item Range over which entry is valid
\item Irradiance Gradients
\end{itemize}
\subsection{Caustics}
Cras nisi neque, pharetra ac cursus nec, vestibulum sit amet erat. Vivamus eget viverra elit. Sed vehicula augue sit amet nibh convallis volutpat. Sed feugiat posuere nunc a auctor. Nam turpis erat, ultrices sed varius in, tempus nec enim. Donec hendrerit dignissim libero, non lacinia odio congue non. Nulla eu velit urna, ut accumsan nibh. Fusce ligula massa, volutpat ut blandit at, dignissim sed orci. 
\subsection{Indirect Illumination}
Cras nisi neque, pharetra ac cursus nec, vestibulum sit amet erat. Vivamus eget viverra elit. Sed vehicula augue sit amet nibh convallis volutpat. Sed feugiat posuere nunc a auctor. Nam turpis erat, ultrices sed varius in, tempus nec enim. Donec hendrerit dignissim libero, non lacinia odio congue non. Nulla eu velit urna, ut accumsan nibh. Fusce ligula massa, volutpat ut blandit at, dignissim sed orci. 
\subsection{Flexibility}
Class aptent taciti sociosqu ad litora torquent per conubia nostra, per inceptos himenaeos. Quisque at arcu risus. Vestibulum ullamcorper quam ut ipsum elementum porta. Proin et lorem nibh, vel vulputate justo. Nulla facilisi. Morbi diam mi, eleifend et euismod eget, blandit ut mi. Cras et convallis erat. 

\chapter{Optimizations}
\section{Overview}
The primary path to efficiency: avoid tracing rays whenever possible; and above all, avoid ray–surface intersection tests.
\section{Finding Bottlenecks}
Profiling tools can be used to find details about execution time. For this purpose Instruments from Apple is used under Mac OS X Lion.
Instruments is a mature system able to show execution of individual threads per core and execution time per function. This is very useful during
optimization and has been used during test and development of the ray tracer described in this report.
\section{Triangle intersection}
Cras nisi neque, pharetra ac cursus nec, vestibulum sit amet erat. Vivamus eget viverra elit. Sed vehicula augue sit amet nibh convallis volutpat. Sed feugiat posuere nunc a auctor. Nam turpis erat, ultrices sed varius in, tempus nec enim. Donec hendrerit dignissim libero, non lacinia odio congue non. 
\subsection{The M\"{o}ller-Trumbore test}
Class aptent taciti sociosqu ad litora torquent per conubia nostra, per inceptos himenaeos. Quisque at arcu risus. Vestibulum ullamcorper quam ut ipsum elementum porta. Proin et lorem nibh, vel vulputate justo. Nulla facilisi. Morbi diam mi, eleifend et euismod eget, blandit ut mi. Cras et convallis erat. Fusce viverra ante eu sapien blandit vitae venenatis dolor rhoncus. Morbi sit amet augue eget ante dignissim vestibulum quis ac ipsum. Morbi eu leo mi, eu varius ipsum. Donec facilisis diam nec leo imperdiet ornare. Sed molestie tortor quis eros pharetra viverra in vel elit. Phasellus ut euismod enim
\cite{moller1997}
\subsection{Barycentric coordinates}
\section{Culling}
Cras nisi neque, pharetra ac cursus nec, vestibulum sit amet erat. Vivamus eget viverra elit. Sed vehicula augue sit amet nibh convallis volutpat. Sed feugiat posuere nunc a auctor. Nam turpis erat, ultrices sed varius in, tempus nec enim. Donec hendrerit dignissim libero, non lacinia odio congue non. Nulla eu velit urna, ut accumsan nibh. Fusce ligula massa, volutpat ut blandit at, dignissim sed orci. 
\subsection{View Frustrum Culling}
Cras nisi neque, pharetra ac cursus nec, vestibulum sit amet erat. Vivamus eget viverra elit. Sed vehicula augue sit amet nibh convallis volutpat. Sed feugiat posuere nunc a auctor. Nam turpis erat, ultrices sed varius in, tempus nec enim. Donec hendrerit dignissim libero, non lacinia odio congue non. 
\subsection{Backface Culling}
Cras nisi neque, pharetra ac cursus nec, vestibulum sit amet erat. Vivamus eget viverra elit. Sed vehicula augue sit amet nibh convallis volutpat. Sed feugiat posuere nunc a auctor. Nam turpis erat, ultrices sed varius in, tempus nec enim. Donec hendrerit dignissim libero, non lacinia odio congue non. 
\section{Tiled rendering}
Subdividing the rendering into a regular grid of tiles of 32x32 pixels, see Section~\ref{ch:results:sec:benchmarks}.
\section{Hierarchical Spatial Subdivision}
Minimize the number of intersection tests to increase performance. Intersection tests are costly.
\subsection{Kd-tree}
For static scenes, kd-trees are regarded as the most efficient acceleration structure\cite{popov07GPURT}.
Time vs depth. (Pharr \& Humphreys, 2004)
\subsubsection{Partitioning}
Pellentesque felis nulla, interdum eu pretium vitae, tincidunt ultrices erat. Integer a quam ut nibh lobortis vulputate. Phasellus felis augue, sagittis vitae tempus porta, placerat ut dui. Quisque commodo, erat ut iaculis viverra, urna dui luctus erat, ac rutrum ligula mauris non arcu.
\subsubsection{Spatial median splitting}
\[
split = \mu_{1/2}(p.a)
\]
\subsubsection{Surface Area Heuristic}
The probability the ray hits region A, given that it hits region B:
\[
P(A|B) = \dfrac{SA(A)}{SA(B)}
\]
SA(V) the surface area of V.
\subsubsection{Tree construction}
Class aptent taciti sociosqu ad litora torquent per conubia nostra, per inceptos himenaeos. Quisque at arcu risus. Vestibulum ullamcorper quam ut ipsum elementum porta. Proin et lorem nibh, vel vulputate justo. Nulla facilisi. Morbi diam mi, eleifend et euismod eget, blandit ut mi. Cras et convallis erat.
\subsubsection{Ray intersection}
Class aptent taciti sociosqu ad litora torquent per conubia nostra, per inceptos himenaeos. Quisque at arcu risus. Vestibulum ullamcorper quam ut ipsum elementum porta. Proin et lorem nibh, vel vulputate justo. Nulla facilisi. Morbi diam mi, eleifend et euismod eget, blandit ut mi. Cras et convallis erat.
\begin{algorithm}
\caption{Calculate $y = x^n$}
\begin{algorithmic} 
\REQUIRE $n \geq 0 \vee x \neq 0$
\ENSURE $y = x^n$
\STATE $y \leftarrow 1$
\IF{$n < 0$}
\STATE $X \leftarrow 1 / x$
\STATE $N \leftarrow -n$
\ELSE
\STATE $X \leftarrow x$
\STATE $N \leftarrow n$
\ENDIF
\WHILE{$N \neq 0$}
\IF{$N$ is even}
\STATE $X \leftarrow X \times X$
\STATE $N \leftarrow N / 2$
\ELSE[$N$ is odd]
\STATE $y \leftarrow y \times X$
\STATE $N \leftarrow N - 1$
\ENDIF
\ENDWHILE
\end{algorithmic}
\end{algorithm}
\section{Multithreading}
Class aptent taciti sociosqu ad litora torquent per conubia nostra, per inceptos himenaeos. Quisque at arcu risus. Vestibulum ullamcorper quam ut ipsum elementum porta. Proin et lorem nibh, vel vulputate justo. Nulla facilisi. Morbi diam mi, eleifend et euismod eget, blandit ut mi. Cras et convallis erat.
\subsection{Exploiting parallelism}
Screen divided into regular blocks of a power of two and assigned one separate thread to each block using a thread pool. Every block
is added to the queue of the thread pool.
\subsection{Thread pool}
\begin{itemize}
\item The thread pool pattern
\item Asynchronous task processing within the same process
\item A number of N threads are created to perform a number of M tasks
\item Typically, N $\ll$ M
\item As soon as a thread completes its task, it will request the next task from the queue
\item The number of threads used (N) is a parameter that can be tuned to provide the best performance
\item Thread overhead is minimized, due to avoidance of create and destroy of new threads every work iteration
\item Queues: todo, competed
\item Maintains a pool of worker threads
\item N is static during execution in this implementation
\item Memory efficient, minimize context switching between threads, tuned parameters
\item Thread-safety of the work and complete queues
\end{itemize}
\section{SIMD-instructions}
Pellentesque felis nulla, interdum eu pretium vitae, tincidunt ultrices erat. Integer a quam ut nibh lobortis vulputate. Phasellus felis augue, sagittis vitae tempus porta, placerat ut dui. Quisque commodo, erat ut iaculis viverra, urna dui luctus erat, ac rutrum ligula mauris non arcu.
\subsection{Intel SSE}
Streaming SIMD Extensions, SSE, is an extension to the Intel x86 architecture.
\subsubsection{Registers}
Pellentesque felis nulla, interdum eu pretium vitae, tincidunt ultrices erat. Integer a quam ut nibh lobortis vulputate. Phasellus felis augue, sagittis vitae tempus porta, placerat ut dui. Quisque commodo, erat ut iaculis viverra, urna dui luctus erat, ac rutrum ligula mauris non arcu.
\subsubsection{Instructions}
Pellentesque felis nulla, interdum eu pretium vitae, tincidunt ultrices erat. Integer a quam ut nibh lobortis vulputate. Phasellus felis augue, sagittis vitae tempus porta, placerat ut dui. Quisque commodo, erat ut iaculis viverra, urna dui luctus erat, ac rutrum ligula mauris non arcu.
\section{Multiple rays}
Cras nisi neque, pharetra ac cursus nec, vestibulum sit amet erat. Vivamus eget viverra elit. Sed vehicula augue sit amet nibh convallis volutpat. Sed feugiat posuere nunc a auctor. Nam turpis erat, ultrices sed varius in, tempus nec enim. Donec hendrerit dignissim libero, non lacinia odio congue non. Nulla eu velit urna, ut accumsan nibh. Fusce ligula massa, volutpat ut blandit at, dignissim sed orci. 
\section{Cache optimizations}
Pellentesque felis nulla, interdum eu pretium vitae, tincidunt ultrices erat. Integer a quam ut nibh lobortis vulputate. Phasellus felis augue, sagittis vitae tempus porta, placerat ut dui. Quisque commodo, erat ut iaculis viverra, urna dui luctus erat, ac rutrum ligula mauris non arcu.
\subsection{Cache coherence}
\chapter{Results}
\label{ch:results}
\section{Benchmark system}
Lorem ipsum dolor sit amet, consectetur adipiscing elit. Donec semper accumsan velit, eget consequat tortor aliquam vitae. Nunc ut nisi metus. Pellentesque risus ipsum, facilisis vel porttitor accumsan, volutpat a massa. Nam id lorem tortor, et mollis felis. Suspendisse potenti. Nunc a turpis ac ipsum cursus malesuada sed quis elit. Sed ante diam, porttitor id consectetur a, condimentum sed purus. Mauris ultricies justo eget ante hendrerit sed auctor lacus lobortis. Curabitur turpis erat, suscipit at ullamcorper ut, fringilla non metus. Cras porttitor ipsum quis libero interdum quis gravida magna sodales. Fusce euismod, tellus vitae porta dictum, nunc tellus tempus turpis, a elementum justo magna vitae metus. Mauris erat felis, posuere bibendum malesuada sit amet, tincidunt vel est. Suspendisse laoreet ullamcorper ligula, et molestie orci ullamcorper ut. Praesent mollis sem a nibh condimentum adipiscing. Aenean vitae magna magna. Praesent quis lorem ipsum, eget lobortis tortor.
\section{Testing scenes}
Pellentesque felis nulla, interdum eu pretium vitae, tincidunt ultrices erat. Integer a quam ut nibh lobortis vulputate. Phasellus felis augue, sagittis vitae tempus porta, placerat ut dui. Quisque commodo, erat ut iaculis viverra, urna dui luctus erat, ac rutrum ligula mauris non arcu. Maecenas nisi nunc, pulvinar in interdum eget, adipiscing quis lorem. Etiam hendrerit faucibus orci, ac elementum dui posuere eget. Vivamus felis purus, eleifend non tempus sed, dapibus in risus. Suspendisse et ipsum augue.
\section{Benchmarks}
Ut in erat tellus. Vestibulum ante ipsum primis in faucibus orci luctus et ultrices posuere cubilia Curae; Integer id fringilla massa. Praesent ut urna vel neque eleifend congue. Donec libero felis, placerat ac scelerisque sed, eleifend ac nisl. Nullam sapien odio, volutpat at venenatis nec, sollicitudin condimentum magna. In hac habitasse platea dictumst. Integer enim turpis, pellentesque in commodo ac, sagittis lobortis erat.
\label{ch:results:sec:benchmarks}
%-----------------------------------------------------------
\addcontentsline{toc}{chapter}{\numberline{}Bibliography}
\bibliographystyle{plain}
\bibliography{biblio}
%-----------------------------------------------------------
\appendix
\chapter{Rendered images}
\section{Cornell box}
\section{Chess scene}
\section{Dragon scene}
\chapter{File formats}
\section{OBJ format}
\section{PLY format}
\section{3Ds format}
\section{TGA format}
%\chapter{Rendered images}
\section{Cornell box}
\section{Chess scene}
\section{Dragon scene}
%\chapter{File formats}
\section{OBJ format}
\section{PLY format}
\section{3Ds format}
\section{TGA format}
%\include{app5}
%\include{app3}
%-----------------------------------------------------------
\end{document}
